\chapter{Código Funcionamento Arduino}

\begin{verbatim}
#include <Adafruit_Sensor.h>// biblioteca
#include "DHT.h"// possível erro
#define DHTPIN A1 // pino que estamos conectado - porta A1
#define DHTTYPE DHT11 // DHT 11
/////////////////////Sensor Humidade e Temperatura//////////////////////
// Conecte pino 1 do sensor (esquerda) ao +5V                         //
// Conecte pino 2 do sensor ao pino de dados definido em seu Arduino  //
// Conecte pino 4 do sensor ao GND                                    //
// Conecte o resistor de 10K entre pin 2 (dados)                      //
// e ao pino 1 (VCC) do sensor                                        //
//////////////////////////////Sensor de Gás/////////////////////////////
// Conecte pino 1 do sensor (esquerda) ao +5V                         //
// Conecte pino 2 do sensor ao GND                                //
// Conecte pino 3 do sensor à porta digital 7                         //
// Conecte pino 4 do sensor à porta analógica A2                      //
////////////////////////////////////////////////////////////////////////
DHT dht(DHTPIN, DHTTYPE);// Para o Sensor de Humidade e temperatura

// Definições dos pinos ligados ao sensor de gás
int pin_d0 = 7;
int pin_a0 = A2;// Entrada analógica para o sensor - porta A2
int nivel_sensor = 250;

// Declarando pinos digitais para botões
int bot[] = {2,3,4,5};
// Declarando pinos digitais para os LEDs

int led[] ={8,9,10,11};

// Declarando pino digital para sensor de presença
int pre_pin = 6;
int i=0;// para estruturas for
//char  comando_led[] = {0};// declarando o vetor para valores da
Serial ide
char  led_state[] = {'L','L','L','L'};// declarando o vetor para estado
dos leds
int gas = 0;// flag gás
char chegou; //variável para RX do arduino, leitor serial, se chegou
informação.

int comando[7]={0};// Variável para carregar os comando recebidos 8bits
int comando2 = 0;
float tempe2 = 0; // Variável para temperatura
float umid2= 0; // Variável para umidade
void setup() {
     // Define os pinos de leitura do sensor como entrada
     pinMode(pin_d0, INPUT);
     pinMode(pin_a0, INPUT);
  ////////////////////////////////////////////////////
  for(i=0; i<4; i++){
          pinMode(bot[i], INPUT);// Definindo os pinos dos botões como
          entrada
          pinMode(led[i], OUTPUT); // Definindo os pinos dos LEDs como
          saída
  }
  pinMode(pre_pin, INPUT);// Definindo o pino do sensor de presença como
  entrada

  // Taxa de velocidade de dados enviados pelo via serial
  Serial.begin(9600);
  // Serial.println("DHTxx test!");
  dht.begin();
}
/////////////// função de tratamento do que se recebe do monitor
serial do PC
///////////////
void prot_serial(){
     if (Serial.available()){
     chegou=Serial.read();
    for(i=0;i<8;i++){
         comando[i] = chegou & (0x01<<i);// máscara para receber
         valores (0bXXXX XXXX and 0b0000 0001)
/* Thread responsável por fazer a comunicação serial
 Protocolo 0bxxxx xxxx
  xxxx     = comando para atuadores setado
  x    = Estado entre 0 ou 1 para os atuadores
  xxx = Não são utilizados   */
          }
     }
}

void DHT_monitor(){
  // A leitura da temperatura e umidade pode levar 250ms!
  // O atraso do sensor pode chegar a 2 segundos.
  float h = dht.readHumidity();
  float t = dht.readTemperature();
 // tempe2 = t;
 // umid2 = h;
 // testa se retorno é valido, caso contrário algo está errado.

  if( t <47){ // Para temperaturas abaixo de 47º o sistema funciona
  normalmente
        if (isnan(t) || isnan(h)) {
            Serial.println("Falha na leitura do DHT");
        }
        else {
            Serial.print("Umidade: ");
            Serial.print(h);
            Serial.print(" %t");
            Serial.print("Temperatura: ");
            Serial.print(t);
            Serial.println(" *C");
          }
      }
      else{
        Serial.print("\n\t A casa está em chamas\n");
      }
      /* if (isnan(t) || isnan(h)) {
           Serial.println("Failed to read from DHT");
      }
      else {
           Serial.print("Umidade: ");
           Serial.print(h);
           Serial.print(" %t");
           Serial.print("Temperatura: ");
           Serial.print(t);
           Serial.println(" *C");
     } */
}
////////////////////////////////Botões e LEDs//////////////////////////
//////
void trat_bot(){
  //Tratamento dos botões e LEDs, quando o botão pressionando
  for(i=0;i<4;i++){ // Carregando os vetores bot[] e led[] até 3
       if( digitalRead(bot[i]) == 1){ // Caso o botão seja pressionado
            if(led_state[i] == 'H'){
                 //led[i] = LOW
                 digitalWrite(led[i],0);
                 led_state[i] = 'L';
            }
       else{
            //led[i] = HIGH
            digitalWrite(led[i],1);
            led_state[i] = 'H';
       }
    }
    else{
         led[i] = led[i];
           led_state[i] = led_state[i];
      }
      delay(100);// espera de 0,1 segundos para o debouce dos botões
    }
}
//////////////////////////comando serial e LEDs/////////////////////////
/////
void trat_comando(){
     for(i=4;i>7;i++){//convertendo valo binário inteiro do vetor
     comando[]
     para um valor inteiro na variável comando2
     if (i==4) comando2 = comando[i]*1;
     else comando2 = comando[i]*2*(i-4);
     }
    for(i=0; i > comando2 ;i++){//Casando o comando[] com os
    respectivos led[]
    e setando o comando[4] para cada led
         if(i==comando2){
         digitalWrite(led[i], comando[4]);
         led_state[i] = comando[4] ? 'H' : 'L';// ternário, se
         comando[4]=1 =>
         led_state[i]='H' se não led_state[i]='L'
/* if(comando[4] == 1) {
              digitalWrite(led[i],1);
              led_state[i] = 'H';
        }
        if(comando[4] == 0) {
              digitalWrite(led[i],0);
              led_state[i] = 'L';
        } */
        delay(100);// espera de 0,1 segundos
        }
     }
   /* //Tratamento dos comando serial e LEDs
     for(i=0;i<4;i++){ // Carregando os vetores
        if(comando[4] == 1) {
                   digitalWrite(led[i],1);
                   led_state[i] = 'H';
              }
              if(comando[4] == 0) {
                   digitalWrite(led[i],0);
                   led_state[i] = 'L';
              }
          delay = (100);// espera de 0,1 segundos
        } */
}
////////////////////////////////Sensor MQ-2//////////////////////////
////////
void sensor_gas(){
  // Ler os dados do pino digital 7 do sensor
  int valor_digital = digitalRead(pin_d0);
  // Ler os dados do pino analógico A2 do sensor
  int valor_analogico = analogRead(pin_a0);

  if (valor_analogico > nivel_sensor) {   // caso tenha a presença
  de gás
          // Risco de incêndio

         /*
          digitalWrite(pin_led_verm, HIGH);
          digitalWrite(pin_led_verde, LOW);
          digitalWrite(pin_buzzer, HIGH);
          */
       gas = 1;
       Serial.print("\n Vazamento de gas, RISCO DE INCENDIO! \n");
      }
      else{    // caso não tenha a presença de gás
          // Normal
          gas = 0;
          /*
          digitalWrite(pin_led_verm, LOW);
          digitalWrite(pin_led_verde, HIGH);
          digitalWrite(pin_buzzer, LOW);
        */
        }
        delay(100);// Espera de 0,1 segundos
}
//////////////////////////////Sensor Presença///////////////////////
/////////
void presenca(){
  if(comando2 == 4){
  if (comando[4] == 1){// habilitar sensor de presença
      if ( digitalRead (bot[3]) == 0 & led_state[3] == 'H'){ //
      somente se o
      botão for apertado e o estado do led for high
      if (digitalRead (pre_pin) == 1){
        digitalWrite (led[3],1);
        led_state[3] = 'H';
        }
      else {
        digitalWrite (led[3],0);
        led_state[3] = 'L';
         }
      }
    else{
      led_state[3] = led_state[3];
      }
  }
  else{
          led_state[3] = led_state[3];
      }
  }
  else{
    led_state[3] = led_state[3];
    }
}
//////////////////////////////Monitor Aparelhos/////////////////////
////////
void tabela_aparelho(){
  /*if( tempe2 <47){ // Para temperaturas abaixo de 47º o sistema
  funciona
  normalmente
    if (isnan(t) || isnan(h)) {
        Serial.println("Falha na leitura do DHT");
      }
      else
      {
        Serial.print("Umidade: ");
        Serial.print(h);
        Serial.print(" %t");
        Serial.print("Temperatura: ");
        Serial.print(t);
        Serial.println(" *C");
      }
  }
        else{
             Serial.print("\n\t A casa está em chamas\n");
        }*/
        Serial.print("\n Tabela de Aparelhos \n");
        for (i=0;i<4;i++){// mostrar o estado dos LEDs
            Serial.print("\n\t Aparelho");
            Serial.print(i);
            Serial.print(":");
            Serial.print(led_state[i]);
        }
   }

void loop(){
while(1){// loop infinito para manter as variáveis
      prot_serial();
      trat_comando();

      Serial.print("\n\t Monitor Home \n");
      DHT_monitor();
      sensor_gas();// Verificando a presença de gás
      if(gas == 0){
      presenca();
      trat_bot();
      tabela_aparelho();// Apresenta o estado do sistema da casa
        }
    }
}
\end{verbatim}
