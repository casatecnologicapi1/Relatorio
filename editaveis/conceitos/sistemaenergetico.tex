\section{Sistema Energético}
\par Aqui iremos descrever os processos e detalhes sobre o sistema e aparelhos energéticos que queremos implantar.

\subsection{Smart Grid}
\par O Smart Grid é um dos pilares deste projeto, pois ele define todos os conceitos de controle autonomia energética propostos.
\par O conceito de Smart Grid que foi descrito por Amin e Wollenberg (2005) é definido como “uma infraestrutura de rede elétrica em larga escala caracterizada por segurança, agilidade, resiliência/robustez que enfrenta novas ameaças e condições não previstas”[Amin e Wollenberg (2005, p.1)]. É um novo sistema de distribuição e transmissão de energia elétrica que funciona com o uso de recursos tecnológicos avançados, baseados em equipamentos digitais integrados. A maior eficiência e a facilidade de gestão energética oferecida pela Smart Grid fazem com que esse sistema seja capaz de atender a diversos tipos de demandas e necessidades exigidas em grandes centros urbanos e industriais.
\par Em resumo, e durante a realização desse projeto, o objetivo das smart grids é otimizar a produção e o consumo de energia na residência onde for aplicada.
Um dos principais componentes desse complexo de redes é o medidor eletrônico inteligente, que substitui o tradicional medidor analógico. Esse equipamento oferece uma série de novas funcionalidades, como o envio de dados em tempo real, medição remota, sistema de alarme de sobrecarga da rede, além dos ajustes automáticos de economia de energia. Tudo isso permite um acompanhamento mais minucioso do consumo, tanto por parte das empresas de energia quanto pela população em geral.
\par Do ponto de vista ambiental, a Smart Grid oferece uma série de benefícios ao ser capaz de reconhecer instantaneamente uma queda no fornecimento da rede e realizar, de modo automático, todas as medidas necessárias para o restabelecimento da distribuição. Essas ações, juntos com um controle mais rígido sobre fraudes e desvios, diminuem bastante o desperdício de energia.  Além disso, as redes inteligentes se integram facilmente com fontes renováveis de energia, tanto para consumidores de baixa tensão (setor residencial) quanto para os de alta tensão (comércios e indústrias).

\subsection{Eficiência Energética}
\par O entendimento de eficiência energética se torna necessário para entender as escolhas feitas para o projeto, assim como a implantação da tecnologia de Smart Grid e os meios alternativos de produção de energia.
\par Temos como eficiente qualquer coisa a qual funciona de acordo com as normas e com o mínimo de erros. Ao projetar o que quer que seja devemos levar em consideração variáveis que vão além dos nossos objetivos. Tomemos como exemplo uma lâmpada: As lâmpadas incandescentes têm o mesmo objetivo das fluorescentes, iluminar o ambiente, entretanto as incandescentes transformam 4\% da energia elétrica em luz e os outros 96\% em energia térmica, percebemos assim sua ineficiência, pois boa parte do recurso foi desperdiçado. Já nas fluorescentes a relação entre produção de luz-calor respectivamente é de 20\% e 80\%, sendo de fato mais eficientes.
\par De acordo com o Plano nacional de eficiência energética (PNEf), eficiência energética são o conjunto de ações de diversas naturezas que culminam na redução de energia necessária para atender as demandas da sociedade por serviços de energia. Têm com objetivo, em síntese, atender às necessidades da economia com menor uso de energia primária e, portanto, menor impacto da natureza
\par No kit para tornar uma casa inteligente estarão inclusos ações básicas que tornam uma casa energeticamente eficiente, como utilização da energia solar como principal fonte de energia, além de pacotes adicionais de implementações que proporcionam um maior aproveitamento dos recursos disponíveis, melhorando assim a eficiência energética da casa. Possibilitando que o consumidor escolha quais pacotes adicionais ele quer contratar, além de recomendações feitas por forma de consultoria visando uma melhoria do consumo e objetivando assim minimizar os gastos energéticos e consequentemente o financeiro.

\subsection{Consumo de Energia}
\par Para nossa base de dados, é indispensável o estudo sobre o consumo de energia, servindo para um comparativo e também para os dados do aplicativo.
\par De acordo com os dados coletados no Anuário Estatístico de Energia Elétrica de 2015, a média de consumo de uma residência no Distrito Federal, dos anos de 2011 à 2015, está em torno de 180,0 kWh/mês, porém, esses dados não levam em consideração o tamanho da habitação ou a quantidade de moradores, além disso, por ser uma média, não considera a classe social da família a qual influencia muito no gasto de energia residencial.
\par Como nosso público alvo são famílias com uma renda familiar acima da média brasileira que é de R\$1.226,00 (IBGE, 2016), foi feita uma estimativa do consumo energético de uma casa de classe média alta, levando em consideração os equipamentos e eletrodoméstico que a casa possui bem como o tempo de uso mensal de cada dispositivo somando ainda o consumo que será acrescentado pela automação e chegou-se a um consumo mensal de aproximadamente 1.100 KWh/mês.

\subsection{Fonte de energia da casa}
\par No projeto para uma maior autonomia da casa, trabalha-se tanto com a energia fornecida comumente, como também com energia de fontes renováveis, que poderão ser implantadas posteriormente.
\par A energia fornecida dita comum são as providas por empresas de distribuição de energia, com CAESB e Light.
\par As fontes de energia renováveis são aquelas que utilizam de recursos renováveis para a geração de energia elétrica.
\par Hoje já existem diversas formas de produzir energia a partir de recursos renováveis, no entanto, a maioria são aplicadas para produção de energia elétrica em grande quantidade. Numa escala residencial, o sistema mais utilizado é a produção de energia a partir de placas solares, pelo fato da alta incidência de raios solares no Brasil, além disso a instalação de sistemas de captura de energia solar fotovoltaica é fácil e rápida, por isso é chamada de plug-and-play (conectar e utilizar, em inglês) e pode-se utilizar a área do telhado para sua instalação. Sua instalação interfere muito pouco no sistema elétrico já existente no imóvel, além de servir de acordo com as necessidades da família. Como o sistema é modular, pode-se instalar um número X de painéis e, caso necessário, mais painéis poderão ser instalados no futuro sem grandes dificuldades.
\par Além de tudo, o investimento inicial na compra de equipamentos para captura de energia solar é rapidamente coberto por meio da economia gerada no futuro, e os custos de manutenção são mínimos. Além disso, no caso da energia solar, a vida útil dos painéis é de 40 anos, o que representa um investimento rentável e duradouro.
\par Por esses fatores, para a residência estudada, optamos em adotar o sistema de placas solares como nossa fonte extra de energia.

\subsection{Como será a utilização da energia solar}
\par Para a utilização da energia provinda das placas será utilizado o sistema on-grid, ou sistema fotovoltaico conectado à rede, no qual possibilita a utilização da energia tanto produzida pelas placa solares quanto a própria energia da rede elétrica da concessionária. Foi escolhido esse sistema pelo fato de não necessitar de dispositivos de armazenamento, pois toda energia produzida em excedente é injetada na rede, no qual é convertido em créditos de energia, que podem ser utilizado em momentos onde a demanda é maior que a produção.
\par O sistema on-grid trabalha em paralelo com a rede pública de distribuição de energia elétrica, ou seja, opera da mesma forma que  uma usina elétrica convencional. A diferença está em sua pequena potência, se comparada a uma grande central de produção. Após a instalação e funcionamento toda a energia gerada é transformada em corrente alternada, por um inversor, para assim ser direcionada para um medidor bidirecional e utilizada pelo no quadro geral da casa e , caso haja, o excedente é injetado na rede.
\par Esse tipo de sistema é regulamentado pela resolução normativa nº 482 da Agência Nacional de Energia Elétrica (ANEEl), de 17 de abril de 2012, que é o que define o mecanismo de compensação de energia. Para a instalação do sistema conectado é necessário solicitar a autorização da distribuidora, mediante a apresentação de um projeto elétrico, um memorial descritivo e outros documentos que comprovem que o sistema segue as as normas vigentes. Ainda, segundo a norma, para fins de compensação, a energia ativa injetada no sistema de distribuição pela unidade consumidora será cedida a título de empréstimo gratuito para a distribuidora, passando a unidade consumidora a ter um crédito em quantidade de energia ativa a ser consumida por um prazo de 60 (sessenta) meses.
\par Para suprir a demanda da casa que foi escolhida para aplicar o projeto, com consumo de cerca de 1100 KWh/mês, são necessárias cerca de 28 placas de 320 Watts de potência. Com isso daria para suprir toda a demanda energética da casa. E nos dias nublados em que as placas possam não suprir a demanda da casa será utilizado a energia elétrica da rede de distribuição. Contudo, mesmo o sistema solar fotovoltaico suprindo todo o consumo de energia elétrica da casa, o valor monetário da conta de luz não chega a zero, pois a distribuidora cobra uma taxa mínima, chamada de custo de disponibilidade, referentes a impostos, encargos setoriais, transmissão pelas usinas  e distribuição pela distribuidora, no caso de brasília, a CEB.

\subsection{Sistema crítico para uma possível queda de energia}
\par O sistema crítico é o conjunto de sensores e atuadores que continuarão funcionando mesmo na falta de energia.
\par Segundo dados da ANEEL (2016), entre 2010 e 2015, houveram em média pouco mais de 10 apagões de energia por ano, totalizando 18,43 horas anuais sem energia elétrica. Tomando estes dados como base, o sistema crítico deve ter autonomia de funcionamento de pelo menos 3 horas.
\par As baterias utilizadas para manter o sistema crítico funcionando vão estar dispostas dentro da casa e será utilizada a própria energia do sistema elétrico da casa para carregar as baterias na qual somente será recarregada após uma queda de energia. 
