\chapter{Metodologia}
    \section{Escolha da Metodologia}
        \par A metodologia escolhida para o gerenciamento do projeto é a proposta pelo Project Management Institute - PMI -, descrita na quinta edição do Guia PMBOK por conta do estilo do tema abordado, criando prazos e escolhas fixas, previamente acordadas para instalação do produto sem nenhum problema.
        \par Para esse projeto dividimos o serviço em setores bem definidos, seguindo uma EAP feita em conjunto.
        \par Cada setor estuda e analisa uma parte do projeto e a descreve, sendo que os subsetores criados a partir dessa união de engenharias também farão estudos das áreas necessárias com integrantes selecionados.
        \par Os setores são áreas de estudo do projeto,  que está descrito e dividido em quatro partes principais:
        \begin{itemize}
            \item Internet das coisas
            \item Projeto da casa
            \item Energia
            \item Circuitos eletrônicos
        \end{itemize}
        \par Dentro dessas quatro partes existem várias interligações que são os subsetores, que serão demonstrados pela EAP.
    	\par Para cada setor temos um responsável técnico totalizando 5 integrantes, cada um com sua equipe, e dois gerentes gerais que facilitam o desenvolvimento do projeto.

    \section{Grupos e Subdivisões}
        \par A equipe foi dividida em 5 grupos. Cada grupo possui um subgerente com o objetivo de facilitar a comunicação e atribuir as atividades referentes ao projeto.
        \begin{itemize}
            \item Gerência de Projeto
                \begin{itemize}
                    \item \textbf{Arthur Guimarães}
                    \item Guilherme Carballal O. Santos
                \end{itemize}
            \item Estrutura e Materiais
                \begin{itemize}
                    \item \textbf{Fulano}
                    \item Gabriel Henrique Chules
                    \item Fulano
                \end{itemize}
            \item Sistema Energético
                \begin{itemize}
                    \item \textbf{Rafael D. Pontes}
                    \item Raiane Pessoa
                    \item Gabriel
                    \item Gabriel
                \end{itemize}
            \item Sistema de Automação
                \begin{itemize}
                    \item \textbf{Fulano}
                    \item Fulano
                    \item Fulano
                \end{itemize}
            \item Desenvolvimento
                \begin{itemize}
                    \item \textbf{Thiago C. Moreira}
                    \item Vitor Bertulucci
                    \item João Pedro Pereira
                \end{itemize}
        \end{itemize}
