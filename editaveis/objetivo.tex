\chapter[Objetivos]{Objetivos}
%\addcontentsline{toc}{chapter}{Introdução}
\section{Objetivo Geral}
\par O presente projeto tem como objetivo geral proporcionar conforto, segurança e economia aos moradores, através da implementação de kits de automação contratados pelos clientes.
\section{Objetivos Específicos}
\par Desenvolver o projeto preliminar de um pacote ou kit de automação residencial que realize o monitoramento dos eletrodomésticos, iluminação e condições do ambiente. O sistema deste kit deve monitorar a casa de forma remota, por meio de um plataforma, em que deve fazer o controle entre os sistemas críticos e não críticos, de forma automática ou configurável, conforme as necessidades do cliente.
\par A casa deve ser adaptada com materiais com custo benefício coerente com a renda do cliente, com o menor impacto negativo ao ambiente, dentro das normas de construção da ABNT.
\par Por meio dos sensores presente no kit, o sistema deve ser capaz de alertar o usuário as condições de risco, em caso de incêndio e furto residencial. Desta forma, o usuário se sente mais seguro, diminuindo suas preocupações rotineiras, evitando acidentes e situações de vulnerabilidade.
\par O sistema deve gerenciar o consumo de energia de acordo com a produção de energia a partir das placas fotovoltaicas, com a necessidade de potência exigida pelos aparelhos da casa ligados em determinado período, com o  consumo de energia da companhia de energia local e com o controle de necessidade aparelhos ligados e desligados. Desta forma o sistema deve proporcionar economia no consumo de energia, diminuindo a conta de luz, se mostrando como um investimento para o cliente.
\par Por meio de uma plataforma, o sistema do kit deve proporcionar ao usuário agilidade e simplicidade nas funções, nas configurações e nas informações obtidas. Assim, o usuário poderá deixar o sistema encarregado para fazer atividades que podem ser configuradas, melhorando a qualidade de vida do usuário.
\par Como forma de proporcionar conforto, segurança e economia, propõe-se o  monitoramento dos eletrodomésticos, iluminação e condições do ambiente, monitoramento  da casa de forma remota,  e fazer adaptações na casa, de forma a torná-la eficiente tanto energeticamente quanto termicamente.
