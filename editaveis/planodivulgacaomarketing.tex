\chapter{Plano de Divulgação e Marketing}

\section{Planejamento}

\subsection{Objetivos e Metas}
\par Os objetivos do nosso plano de divulgação e marketing facilitar a inserção e sucesso do produto no segmento de mercado atuante, cativando o público alvo e facilitando sua aceitação do produto.

\section{Análise de Segmento de Mercado}

\subsection{Microambiente}
\par Detalhes sobre os serviços prestados, variedade do produto, fornecedores, entre outros fatores relacionados ao microambiente, podem ser encontrados na Seção \ref{visaogeral} - Visão Geral localizada na página \pageref{visaogeral}.

\subsection{Macroambiente}
\par O macroambiente é tendenciosamente favorável ao produto visto que o mercado tecnológico cresce no Brasil, acima da média nacional. A demanda nos setores relacionados à tecnologia só tende a crescer e o mercado de serviços de automação ainda se encontra com bastante vagas para novas empresas e produtos.

\subsection{Concorrentes Conhecidos}

\subsubsection{Schneider Electric}
\par A Schneider Electric é uma empresa de Internet of Things e gestão de energia. Desenvolve tecnologias e soluções conectadas para gerenciar energia e processos de maneira segura, confiável, eficiente e sustentável. A empresa investe em P&D, a fim de sustentar a inovação e a diferenciação, com um forte compromisso com o desenvolvimento sustentável.

\subsubsection{Alltok soluções}
\par Empresa voltada para a automação residencial e predial, focada na área de Internet of Things utilizada para controle de eletrônicos , iluminação e o sistema de segurança.

\subsubsection{Arquitectar}
\par Arquitectar é uma empresa voltada para à automação residencial, predial e corporativa. Distribuição e instalação de sistemas de automação, tecnologias residenciais, áudio e vídeo e segurança. Possuindo tecnologia inovadora de controle e outros sistemas como o “PARE” (Programa Automatizado de Recuperação de Energia), recolhendo dados e otimizando o sistema de consumo de energia e hidráulica, também participando do mercado de energia renovável (solar).

\subsubsection{Federal Multi}
\par A Federal Multi atua nos mercados corporativo e governamental desde 2010, com foco nas áreas de automação predial, instalações técnicas especiais e manutenção predial na indústria da construção civil. Voltada principalmente para as áreas de controle de iluminação, eficiência energética, controle de sistemas hidráulicos e aparelhos eletrônicos.

\subsubsection{Starvai}
\par Empresa atuante no mercado desde 2009 voltada para a área de automação residencial, projetando e aplicando suas soluções de acordo com o cliente com um pacote variável focado nas áreas de segurança, iluminação,eletrônicos, áudio e vídeo.

\section{Público Alvo}
\par O segmento escolhido como público alvo do produto são famílias de classe média-alta e classe alta. Os motivos por trás desta escolha são vários: poder aquisitivo; maior contato com recursos tecnológicos e maior interesse quando se trata de casas com tecnologia de automação. Por conta desses motivos a equipe de marketing e divulgação definiu como público alvo este segmento social, visto que esta será a base de consumo o produto quando bem sucedido e maduro no mercado.

\section{Estratégias de Divulgação e Marketing}

\subsection{Comunicação com o Cliente}
\par Serão elaboradas estratégias de divulgação da solução por duas frentes: Marketing Digital, por meio de um website próprio acrescido de divulgação e destaque em ferramentas de busca como o Google, e organização de eventos.
\par A divulgação digital se dará por meio de:
\begin{itemize}
    \item Website contendo maiores informações sobre o produto e seus benefícios, será também definida a identidade visual da empresa e produtos. O website é uma das formas de ganhar maior audiência quanto à público e será constantemente aprimorado de acordo com a análise de uso dos visitantes utilizando o Google Analytics;
    \item Página no Facebook, com atualizações periódicas, informações sobre produtos e inovações;
    \item Marketing de motores de busca, a partir da divulgação do site e produtos com Google Ads, palavras chaves e destaque nas buscas no Google Search;
    \item Mídias sociais onde serão divulgados, junto do website, nossas soluções, em cases além de pequenas propagandas de marketing.
\end{itemize}
\par A organização de eventos menores e presenciais, visa adquirir experiência através de conversas com clientes e novos consumidores. Os conteúdos desses eventos vão desde apresentações do serviço em si, até mesmo esclarecimentos a respeito dos benefícios e funcionalidades dos serviços. Também deve se destacar, o investimento em infraestrutura para esses eventos, gerando maior conforto e satisfação aos clientes.

\subsection{Plano de Retenção de Clientes}
\par O investimento em relacionamento entre a empresa e o cliente é fundamental para a retenção dos mesmos, especialmente se analisarmos o fato de que o público alvo exige cada vez mais a qualidade e o compromisso dos serviços prestado, tendo um bom relacionamento com o cliente.
\par O comprometimento, os eventos, e atendimentos especializados, farão com que os clientes se sintam únicos, trazendo a confiança e comodidade do mesmo, e aumentando as chances de uma nova realização de serviços com a mesma empresa.

\subsection{Plano de Indicação de Clientes}
\par O comprometimento e o relacionamento com os clientes são importantes para que os mesmos, se sintam confiantes para indicar o serviço para seus contatos, gerando novos consumidores, e até fãs da empresa e seus respectivos serviços, além de oferecer prêmios e serviços adicionais para clientes que indicarem, como descontos e outros.
\par Indicações por meio de terceiros e empresas parceiras, também são fundamentais para que se tenha um bom networking e uma vasta rede de contatos.
