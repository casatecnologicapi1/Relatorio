\chapter{Protótipo}

\section{Sobre a Prototipagem}
\par Um protótipo é comumente definido como uma versão preliminar de algum projeto, seja um projeto de carro, software ou construções, sendo que o mesmo deve portar características correspondentes com a proposta e finalidade do projeto.
\par A prototipagem em si possui no mínimo duas etapas que culminam no desenvolvimento dos protótipos de baixa e alta fidelidade.
\par Durante a primeira etapa é feita a concepção das características do produto e a elaboração de um esboço de como o produto deve se apresentar ou se comportar. Ao produto gerado pela primeira etapa damos o nome de protótipo de baixa fidelidade.
\par Durante a segunda etapa é feito o aprofundamento do protótipo de baixa fidelidade, geralmente com o feedback dos stakeholders, sendo que este novo protótipo, chamado de protótipo de alta fidelidade, deve ser o mais parecido possível com o produto final. Porém não é necessário que o mesmo seja feito utilizando as mesmas tecnologias do produto final.
\par A prototipagem de um software é uma técnica bastante utilizada para o entendimento dos requisitos e para a validação do projeto com um cliente.
\par Para um melhor entendimento dos requisitos do software controlador da automação residencial, um protótipo de alta fidelidade foi criado pela equipe responsável pela plataforma.

\section{Sobre o Protótipo do Software}
\par O protótipo de alta fidelidade do software será um dos artefatos desenvolvidos pela equipe para apresentação à banca avaliadora. O protótipo tem como propósito ilustrar como será implementada a interface entre o usuário e todo o sistema de automação residencial.
\par Neste protótipo serão elaboradas as telas mais importantes da aplicação, sendo que estas possuem como finalidade apenas a demonstração da interface.
\par Devido ao prazo, escopo e dificuldade do projeto, a equipe optou por realizar a prototipagem do software como um protótipo de alta fidelidade, complementando os outros protótipos. As funções simuladas pelo protótipo serão apresentadas com maior enfoque na próxima seção deste documento.

\section{Funções Simuladas pelo Protótipo do Software}
\par O protótipo tem como finalidade principal a simulação de ações do usuário em sua interação com o software. O protótipo consistirá de representações gráficas, capazes de interação, das telas de interface do software com o usuário.
\par Assim, o protótipo será capaz de mostrar de forma eficiente como se dará a transmissão de informações entre o sistema, com todos os seus dados coletados por sensores, e o usuário do sistema.

\section{Sobre o Protótipo do Hardware}
\par O protótipo tem como objetivo mostrar o operação de cada tipo de sensor empregado ao projeto, que são  o sensor de umidade e temperatura, gases inflamáveis e fumaça, sensor de presença isso implementados ao microcontrolador Arduino. O arduino está conectado via USB à Raspberry, em que a Raspberry com o sistema operacional Raspbian instalado com a ide do arduino instalado. A versão do Arduino utilizada é a versão UNO R2. Para poder simular as câmeras de segurança, é utilizado uma webcam, conectada via USB, o drive da webcam também está instalado no sistemas da Raspberry. Para representar os aparelhos e iluminação conectados ao sistemas, são utilizados LEDs, porque a forma como se comporta aos comando do Arduino é o mesma. Para programar o Arduino é utilizado a IDE do arduino , programando em em  linguagem C++ para a plataforma Arduino, utilizando as bibliotecas disponível para o uso dos sensores analógicos.  Para se comunicar com o arduino, é utilizado o monitor serial da IDE, assim pode-se enviar comando e receber informações via serial.
\par Para simular a comunicação SSH, é utilizado o VNC, que tem função idêntica, porém depende do servidor que fornece o programa do VNC. Para testar o acesso remoto pela rede, é utilizado um computador que tenha o VNC instalado, independente do sistemas sistema operacional desta máquina.
\par O material utilizado no protótipo é:
\begin{itemize}
    \item 1 Raspberry;
    \item 1 Arduino UNO R2;
    \item 1 sensor de temperatura e umidade DHT11;
    \item 1 sensor de gases tóxicos MQ-2;
    \item 1 sensor de presença Qualitronix QA25;
    \item 4 botões de pressão.
    \item 6 LEDs
\end{itemize}
