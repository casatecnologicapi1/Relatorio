\chapter{Visão Geral}
    \section{Perspectiva do Produto}
        \par O produto visa atender qualquer casa que possua os requisitos mínimos para cada pacote previamente decidido, direcionado a clientes dispostos a alterar sua casa em prol da economia e praticidade, e também atender ao maior número de pessoas. Com isso, teria-se uma  possível melhora na produção e economia energética, tanto do cliente quanto da cidade, e um controle maior da residência.

        \subsection{Quais são os pacotes}
            \par O pacote principal contém a tecnologia Smart Grid e toda a ideia de IoT, tendo o aplicativo base e todos os sensores por nós listados, atendendo a todos os cômodos especificados , cozinha, sala, escritório, quartos, banheiro, garagem e área de serviço.
	        \par Os outros pacotes são compostos por variações dos cômodos especificados e decididos pelo cliente, e quais automações ele gostaria de implementar.

        \subsection{Pacotes Adicionais}
            \par Os pacotes adicionais visam implantar outros requisitos não especificados no pacote principal, como alternativas sustentáveis e elementos ou sensores extras, baseado no que foi proposto.
	        \par Como o objetivo do projeto é uma casa inteligente, e não sustentável, foi decidido que as implementações feitas para tornar a casa mais auto sustentável seriam disponibilizadas na forma de pacotes adicionais, no qual o cliente pode escolher quais pacotes ele quer contratar, juntamente com o projeto. Para a adesão do cliente aos pacotes adicionais seriam feitas recomendações por consultoria, apresentando os benefícios que o cliente vai adquirir com a contratação desses serviços, visando assim uma melhoria no consumo energético da casa, ou reaproveitamento de resíduos.
            \par O cliente também teria a opção de trocar os materiais da casa para torná-la ainda mais eficiente, seja em temperatura, ou energia.
            \par Os sensores extras ou elementos extras também seriam feito por consultoria visando atender também exigências do cliente caso seja possível.

        \subsection{Perspectiva do cliente}
            \par O público alvo do nosso kit são consumidores de classe média que estejam interessados em automatização da casa com o objetivo de ter mais comodidade e menos gastos. Para atingir este público serão feitas divulgações em redes sociais e possíveis parcerias com comerciantes locais de lojas de elétricas e eletrônicos para disponibilização dos equipamentos, além de um site da marca para divulgação.
